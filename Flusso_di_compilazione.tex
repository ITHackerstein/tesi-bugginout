% LTeX: language=it

\chapter{Flusso di compilazione}
\label{chap:flusso-di-compilazione}

Nel seguente capitolo si affronteranno, separatamente, le fasi di compilazione di un programma scritto in BugginOut. \`E bene notare che ciascuna di queste fasi prende in ingresso il risultato della fase precedente.

\section{Analisi lessicale}
\label{sec:analisi-lessicale}

Da \cite{alfred2007compilers}:

\begin{parcolumns}[colwidths={1=0.44\textwidth,2=0.44\textwidth},rulebetween=true,nofirstindent=true,sloppy=true]{2}
	% LTeX: language=en_us
	\colchunk{
		\leftskip=1em
		``The main task of the lexical analyzer is to read the input characters of the \emph{source} program, group them into \emph{lexemes}, and produce as output a sequence of \emph{tokens} for each lexeme in the source program.

		[\ldots]

		When discussing lexical analysis, we use three related but distinct terms:
		\begin{itemize}
			\item A \emph{token} is a pair consisting of a token \emph{name} and an optional attribute \emph{value}. The token name is an abstract symbol representing a kind of lexical unit, e.g., a particular keyword, or a sequence of input characters denoting an identifier. [\ldots]
			\item A \emph{pattern} is a description of the form that the lexemes of a token may take. In the case of a kewyord as a token, the pattern is just the sequence of characters that form the keyword. For identifiers and some other tokens, the pattern is a more complex structure that is matched by many strings.
			\item A \emph{lexeme} is a sequence of characters in the source program that matches the pattern for a token and is identified by the lexical analyzer as an instance of that token.
		\end{itemize}
	}
	% LTeX: language=it
	\colchunk{
		\leftskip=1em
		``Il compito principale di un analizzatore lessicale \`e la lettura dei caratteri di input del programma \emph{sorgente}, raggrupparli in \emph{lessemi} e produrre, come risultato, una sequenza di \emph{token} per ogni lessema del programma sorgente.

		[\ldots]

		Durante la discussione dell'analisi lessicale, si usano tre legati ma distinti termini:
		\begin{itemize}
			\item Un \emph{token} \`e una coppia costituita di un \emph{nome} e un \emph{valore} opzionale. Il nome \`e un simbolo astratto che rappresenta il tipo di unit\`a lessicale, ad esempio, una particolare parola chiave oppure una sequenza di caratteri che denota un identificatore. [\dots]
			\item Un \emph{pattern} \`e una descrizione della forma che i lessemi di un token possono avere. Nel caso di una parola chiave come token, il pattern \`e semplicemente la sequenza di caratteri che formano la parola chiave. Per gli identificatori e qualche altro token, il pattern \`e una struttura pi\`u complessa che riconosce pi\`u stringhe.
			\item Un \emph{lessema} \`e una sequenza di caratteri del programma sorgente.''
		\end{itemize}
	}
	\colplacechunks
\end{parcolumns}

In BugginOut, i token utilizzati sono diversi, alcuni dei quali sono:
\begin{table}[H]
	\begin{tabularx}{\textwidth}{X|X|X}
		\hline
		\hline
		\multicolumn{1}{c|}{\textsc{Token}} & \multicolumn{1}{c|}{\textsc{Pattern} (informale)} & \multicolumn{1}{c}{\textsc{Lessemi}} \\
		\hline
		\texttt{kw\_for} & i seguenti caratteri: \texttt{for} & \texttt{for} \\
		\hline
		\texttt{kw\_if} & i seguenti caratteri: \texttt{if} & \texttt{if} \\
		\hline
		\texttt{Identifier} & una lettera oppure \mbox{\texttt{\_} o \texttt{\$}} seguita da una serie di lettere, cifre numeriche oppure \mbox{\texttt{\_} o \texttt{\$}} & \texttt{a}, \texttt{b}, \texttt{token\_name}, \texttt{num1}, \ldots \\
		\hline
		\texttt{IntegerLiteral} & uno o pi\`u cifre decimali\footnotemark & \texttt{1}, \texttt{123}, \ldots \\
		\hline
		\hline
	\end{tabularx}
	\caption{Alcuni token di BugginOut}
	\label{fig:bugginout-example-tokens}
\end{table}

\footnotetext{In BugginOut \`e possibile utilizzare i numeri in notazione binari, ottale ed esadecimale. Per esempio \texttt{0b101} \`e un numero binario, \texttt{0o127} \`e un numero ottale e \texttt{0x1A} \`e un numero esadecimale. Inoltre possono essere seguiti da un suffisso per indicarne il tipo. Il pattern per entrambi \`e stato omesso in quanto non rilevante per l'esempio.}

Nella fase di analisi lessicale si alternano due operazioni:
\begin{itemize}
	\item definita da \cite{alfred2007compilers} come \emph{scanning}, si ignorano i caratteri che non sono significativi per il linguaggio (spazi bianchi e \emph{commenti});
	\item definita da \cite{alfred2007compilers} come \emph{lexical analysis}, \`e la parte pi\`u complessa in cui si producono i token.
\end{itemize}

Quando pi\`u di un lessema pu\`o essere riconosciuto da un pattern, l'analizzatore lessicale inserisce nel valore delle informazioni aggiuntive utili alle fasi successive di compilazione. Ad esempio, nel caso di un numero, il valore \`e il numero stesso. Ad esempio, facendo riferimento all'esempio \ref{fig:bugginout-example-tokens}, il token \texttt{IntegerLiteral} conterrebbe come valore \texttt{123}.

\`E importante osservare che ci\`o che influenzer\`a le decisioni di analisi grammaticale del compilatore \`e il nome del token e non il suo valore che, invece, viene usato per la traduzione dei token.

Per completezza, di seguito verranno riportati tutti i token definiti in BugginOut con pattern denotato con la sintassi POSIX ERE (vedi \cite{iso-9945-2009}).

\begin{xltabular}{\textwidth}{l|X}
	\caption{Token di BugginOut}
	\label{fig:bugginout-complete-tokens} \\

	\hline
	\hline
	\multicolumn{1}{c|}{\textsc{Token}} & \multicolumn{1}{c|}{\textsc{Pattern}} \\
	\hline
	\endfirsthead

	\hline
	\multicolumn{1}{c|}{\textsc{Token}} & \multicolumn{1}{c|}{\textsc{Pattern}} \\
	\hline
	\endhead

	\hline
	\endfoot

	\hline
	\hline
	\endlastfoot

	\texttt{kw\_anon} & \texttt{anon} \\ \hline
	\texttt{kw\_as} & \texttt{as} \\ \hline
	\texttt{kw\_bool} & \texttt{bool} \\ \hline
	\texttt{kw\_break} & \texttt{break} \\ \hline
	\texttt{kw\_char} & \texttt{char} \\ \hline
	\texttt{kw\_continue} & \texttt{continue} \\ \hline
	\texttt{kw\_else} & \texttt{else} \\ \hline
	\texttt{kw\_f32} & \texttt{f32} \\ \hline
	\texttt{kw\_f64} & \texttt{f64} \\ \hline
	\texttt{kw\_false} & \texttt{false} \\ \hline
	\texttt{kw\_fn} & \texttt{fn} \\ \hline
	\texttt{kw\_for} & \texttt{for} \\ \hline
	\texttt{kw\_i16} & \texttt{i16} \\ \hline
	\texttt{kw\_i32} & \texttt{i32} \\ \hline
	\texttt{kw\_i64} & \texttt{i64} \\ \hline
	\texttt{kw\_i8} & \texttt{i8} \\ \hline
	\texttt{kw\_if} & \texttt{if} \\ \hline
	\texttt{kw\_in} & \texttt{in} \\ \hline
	\texttt{kw\_isize} & \texttt{isize} \\ \hline
	\texttt{kw\_mut} & \texttt{mut} \\ \hline
	\texttt{kw\_null} & \texttt{null} \\ \hline
	\texttt{kw\_return} & \texttt{return} \\ \hline
	\texttt{kw\_true} & \texttt{true} \\ \hline
	\texttt{kw\_u16} & \texttt{u16} \\ \hline
	\texttt{kw\_u32} & \texttt{u32} \\ \hline
	\texttt{kw\_u64} & \texttt{u64} \\ \hline
	\texttt{kw\_u8} & \texttt{u8} \\ \hline
	\texttt{kw\_usize} & \texttt{usize} \\ \hline
	\texttt{kw\_var} & \texttt{var} \\ \hline
	\texttt{kw\_void} & \texttt{void} \\ \hline
	\texttt{Ampersand} & \texttt{\&} \\ \hline
	\texttt{AmpersandEquals} & \texttt{\&=} \\ \hline
	\texttt{Asterisk} & \texttt{*} \\ \hline
	\texttt{AsteriskEquals} & \texttt{*=} \\ \hline
	\texttt{At} & \texttt{@} \\ \hline
	\texttt{BinaryLiteral} & \texttt{0b[01]+(\_[a-zA-Z\_\$][a-zA-Z\_\$0-9]*)?}  \\ \hline
	\texttt{CharLiteral} & \texttt{\textquotesingle([\textasciicircum \textquotesingle \textbackslash \textbackslash]|\textbackslash \textbackslash \textquotesingle |\textbackslash \textbackslash n|\textbackslash \textbackslash r|\textbackslash \textbackslash t|\textbackslash\textbackslash\textbackslash\textbackslash|\textbackslash \textbackslash 0|\textbackslash \textbackslash x[0-7][0-9a-fA-F])\textquotesingle} \\ \hline
	\texttt{Circumflex} & \texttt{\textbackslash\textasciicircum} \\ \hline
	\texttt{CircumflexEquals} & \texttt{\textbackslash\textasciicircum=} \\ \hline
	\texttt{Colon} & \texttt{:} \\ \hline
	\texttt{Comma} & \texttt{,} \\ \hline
	\texttt{DecimalLiteral} & \texttt{[0-9]+(\_[a-zA-Z\_\$][a-zA-Z\_\$0-9]*)?} \\ \hline
	\texttt{Dot} & \texttt{\textbackslash .} \\ \hline
	\texttt{DotDotEquals} & \texttt{\textbackslash .\textbackslash .=} \\ \hline
	\texttt{DotDotLessThan} & \texttt{\textbackslash .\textbackslash .<} \\ \hline
	\texttt{DoubleAmpersand} & \texttt{\&\&} \\ \hline
	\texttt{DoubleAmpersandEquals} & \texttt{\&\&=} \\ \hline
	\texttt{DoubleEquals} & \texttt{==} \\ \hline
	\texttt{DoublePipe} & \texttt{\textbackslash |\textbackslash |} \\ \hline
	\texttt{DoublePipeEquals} & \texttt{\textbackslash |\textbackslash |=} \\ \hline
	\texttt{EndOfFile} & Non ha un pattern associato in quanto viene generato quando l'analizzatore lessicale ha terminato di analizzare il programma sorgente. \\ \hline
	\texttt{Equals} & \texttt{=} \\ \hline
	\texttt{ExclamationMark} & \texttt{!} \\ \hline
	\texttt{ExclamationMarkEquals} & \texttt{!=} \\ \hline
	\texttt{FloatLiteral} & \texttt{[0-9]+\textbackslash.[0-9]+(\_[a-zA-Z\_\$][a-zA-Z\_\$0-9]*)?} \\ \hline
	\texttt{GreaterThan} & \texttt{>} \\ \hline
	\texttt{GreaterThanEquals} & \texttt{>=} \\ \hline
	\texttt{HexadecimalLiteral} & \texttt{0x[0-9a-fA-F]+(\_[a-zA-Z\_\$][a-zA-Z\_\$0-9]*)?} \\ \hline
	\texttt{Identifier} & \texttt{[a-zA-Z\_\$][a-zA-Z\_\$0-9]*} \\ \hline
	\texttt{LeftCurlyBracket} & \texttt{\{} \\ \hline
	\texttt{LeftParenthesis} & \texttt{\textbackslash (} \\ \hline
	\texttt{LeftShift} & \texttt{<<} \\ \hline
	\texttt{LeftShiftEquals} & \texttt{<<=} \\ \hline
	\texttt{LeftSquareBracket} & \texttt{\textbackslash \char"5B} \\ \hline
	\texttt{LessThan} & \texttt{<} \\ \hline
	\texttt{LessThanEquals} & \texttt{<=} \\ \hline
	\texttt{Minus} & \texttt{-} \\ \hline
	\texttt{MinusEquals} & \texttt{-=} \\ \hline
	\texttt{MinusMinus} & \texttt{--} \\ \hline
	\texttt{OctalLiteral} & \texttt{0o[0-7]+(\_[a-zA-Z\_\$][a-zA-Z\_\$0-9]*)?} \\ \hline
	\texttt{Percent} & \texttt{\%} \\ \hline
	\texttt{PercentEquals} & \texttt{\%=} \\ \hline
	\texttt{Pipe} & \texttt{\textbackslash |} \\ \hline
	\texttt{PipeEquals} & \texttt{\textbackslash |=} \\ \hline
	\texttt{Plus} & \texttt{+} \\ \hline
	\texttt{PlusEquals} & \texttt{+=} \\ \hline
	\texttt{PlusPlus} & \texttt{++} \\ \hline
	\texttt{RightCurlyBracket} & \texttt{\}} \\ \hline
	\texttt{RightParenthesis} & \texttt{\textbackslash )} \\ \hline
	\texttt{RightShift} & \texttt{>>} \\ \hline
	\texttt{RightShiftEquals} & \texttt{>>=} \\ \hline
	\texttt{RightSquareBracket} & \texttt{\textbackslash ]} \\ \hline
	\texttt{Semicolon} & \texttt{;} \\ \hline
	\texttt{Solidus} & \texttt{/} \\ \hline
	\texttt{SolidusEquals} & \texttt{/=} \\ \hline
	\texttt{StringLiteral} & \texttt{\textquotedbl ([\textasciicircum \textquotedbl \textbackslash \textbackslash]|\textbackslash \textbackslash \textquotedbl|\textbackslash \textbackslash n|\textbackslash \textbackslash r|\textbackslash \textbackslash t|\textbackslash \textbackslash \textbackslash \textbackslash|\textbackslash \textbackslash 0|\textbackslash \textbackslash x[0-7][0-9a-fA-F])*\textquotedbl} \\ \hline
	\texttt{Tilde} & \texttt{\textasciitilde}
\end{xltabular}

Per riconoscere i token, si passa dai pattern ad una rappresentazione intermedia definita da \cite{alfred2007compilers} come \emph{transition diagram} (in altre parole un automa a stati finiti). Analizziamo, ad esempio, come questo viene fatto per il token \texttt{Identifier}.
\begin{figure}[H]
	\centering
	\begin{tikzpicture}[node distance=3.5cm, on grid, auto]
		\node[state, initial] (0) {$q_0$};
		\node[state, accepting] (1) [right=of 0] {$q_1$};

		\path[->]
		(0) edge node {\texttt{[a-zA-Z\_\$]}} (1)
		(1) edge [loop above] node {\texttt{[a-zA-Z\_\$0-9]}} ();
	\end{tikzpicture}
	\caption{Diagramma di transizione per il token \texttt{Identifier}}
	\label{fig:bugginout-identifier-transition-diagram}
\end{figure}

Questo diagramma di transizione \`e composto da due stati: lo stato iniziale $q_0$ e lo stato finale $q_1$. Quando il diagramma \`e in uno stato finale, significa che il token \`e stato riconosciuto. In questo caso, il diagramma inizia dallo stato $q_0$ e si sposta nello stato finale $q_1$ quando incontra un carattere che corrisponde al pattern. Da questo punto in poi, il diagramma rimane nello stato finale fino a quando non incontra un carattere che non corrisponde al pattern.

Un esempio pi\`u complesso \`e quello del riconoscimento dei numeri (i token \texttt{BinaryLiteral}, \texttt{OctalLiteral}, \texttt{HexadecimalLiteral}, \texttt{DecimalLiteral} e \texttt{FloatLiteral}):
\begin{figure}[H]
	\centering
	\scalebox{0.6}{\begin{tikzpicture}[node distance=3.5cm, on grid, auto]
		\node[state, initial] (0) {$q_0$};
		\node[state] (1) [right=of 0] {$q_1$};
		\node[state] (2) [right=of 1] {$q_2$};
		\node[state] (3) [below=of 2] {$q_3$};
		\node[state] (4) [below=of 3] {$q_4$};
		\node[state, accepting] (5) [right=of 2] {$q_5$};
		\node[state] (6) [right=of 5] {$q_6$};
		\node[state, accepting] (7) [right=of 6] {$q_7$};
		\node[state, accepting] (8) [right=of 3] {$q_8$};
		\node[state] (9) [right=of 8] {$q_9$};
		\node[state, accepting] (10) [right=of 9] {$q_{10}$};
		\node[state, accepting] (11) [right=of 4] {$q_{11}$};
		\node[state] (12) [right=of 11] {$q_{12}$};
		\node[state, accepting] (13) [right=of 12] {$q_{13}$};
		\node[state, accepting] (14) [below=10.5cm of 1] {$q_{14}$};
		\node[state] (15) [right=of 14] {$q_{15}$};
		\node[state, accepting] (16) [right=of 15] {$q_{16}$};
		\node[state] (17) [right=of 16] {$q_{17}$};
		\node[state, accepting] (18) [right=of 17] {$q_{18}$};
		\node[state] (19) [below=of 15] {$q_{19}$};
		\node[state, accepting] (20) [right=of 19] {$q_{20}$};

		\path[->]
		(0) edge node {\texttt{0}} (1)
		(1) edge node {\texttt{b}} (2)
		(1) edge node {\texttt{o}} (3)
		(1) edge node {\texttt{x}} (4)
		(1) edge node {\texttt{[0-9]}} (14)
		(2) edge node {\texttt{[01]}} (5)
		(5) edge [loop above] node {\texttt{[01]}} ()
		(5) edge node {\texttt{\_}} (6)
		(6) edge node {\texttt{[a-zA-Z\_\$]}} (7)
		(7) edge [loop above] node {\texttt{[a-zA-Z\_\$0-9]}} ()
		(3) edge node {\texttt{[0-7]}} (8)
		(8) edge [loop above] node {\texttt{[0-7]}} ()
		(8) edge node {\texttt{\_}} (9)
		(9) edge node {\texttt{[a-zA-Z\_\$]}} (10)
		(10) edge [loop above] node {\texttt{[a-zA-Z\_\$0-9]}} ()
		(4) edge node {\texttt{[0-9a-fA-F]}} (11)
		(11) edge [loop above] node {\texttt{[0-9a-fA-F]}} ()
		(11) edge node {\texttt{\_}} (12)
		(12) edge node {\texttt{[a-zA-Z\_\$]}} (13)
		(13) edge [loop above] node {\texttt{[a-zA-Z\_\$0-9]}} ()
		(0) edge node {\texttt{[1-9]}} (14)
		(14) edge [loop left] node {\texttt{[0-9]}} ()
		(14) edge node {\texttt{.}} (15)
		(15) edge node {\texttt{[0-9]}} (16)
		(16) edge [loop above] node {\texttt{[0-9]}} ()
		(16) edge node {\texttt{\_}} (17)
		(17) edge node {\texttt{[a-zA-Z\_\$]}} (18)
		(18) edge [loop above] node {\texttt{[a-zA-Z\_\$0-9]}} ()
		(14) edge node {\texttt{\_}} (19)
		(19) edge node {\texttt{[a-zA-Z\_\$]}} (20)
		(20) edge [loop above] node {\texttt{[a-zA-Z\_\$0-9]}} ();
	\end{tikzpicture}}
	\caption{Diagramma di transizione per i token numerici}
	\label{fig:bugginout-number-transition-diagram}
\end{figure}

In questo caso:
\begin{itemize}
	\item $q_5$ e $q_7$ sono gli stati finali per \texttt{BinaryLiteral};
	\item $q_8$ e $q_{10}$ sono gli stati finali per \texttt{OctalLiteral};
	\item $q_{11}$ e $q_{13}$ sono gli stati finali per \texttt{HexadecimalLiteral};
	\item $q_{14}$ e $q_{20}$ sono gli stati finali per \texttt{DecimalLiteral};
	\item $q_{16}$ e $q_{18}$ sono gli stati finali per \texttt{FloatLiteral}.
\end{itemize}

Si possono costruire diagrammi di transizione per ogni token e, inoltre, \`e possibile unirli per averne uno solo che rappresenter\`a l'intero funzionamento dell'analizzatore lessicale.

\section{Analisi grammaticale}
\label{sec:analisi-grammaticale}

L'analisi grammaticale costituisce la fase in cui la sequenza di token prodotta dall'analisi lessicale viene organizzata secondo le regole sintattiche del linguaggio, definite da una \emph{grammatica} formale. Lo scopo di questa fase \`e costruire una struttura ad albero, l'\emph{AST} (\textit{Abstract Syntax Tree}), che rappresenta gerarchicamente la struttura del programma. Ogni nodo di questo albero rappresenta un costrutto sintattico del linguaggio, come espressioni, dichiarazioni o blocchi di codice. Compito di questa fase \`e anche individura gli \emph{errori sintattici} come, ad esempio, parentesi non bilanciate o costrutti mal formati.

Per definire il comportamento dell'analizzatore grammaticale \`e necessario definire la grammatica del linguaggio definita da \cite{alfred2007compilers}:
\begin{parcolumns}[colwidths={1=0.44\textwidth,2=0.44\textwidth},rulebetween=true,nofirstindent=true,sloppy=true]{2}
	% LTeX: language=en_us
	\colchunk{
		\leftskip=1em
		``A context-free grammar (grammar for short) consists of \emph{terminals}, \emph{nonterminals}, a \emph{start symbol} and \emph{productions}.

		\begin{itemize}
			\item \emph{Terminals} are the basic symbols from which strings are formed. The term "token name" is a synonim for "terminal" [\ldots].
			\item \emph{Nonterminals} are syntactic variables that denote sets of strings. [\dots]. Nonterminals impose a hierarchical structure on the language that is key to syntax analysis and translation.
			\item In a grammar, one nonterminal is distinguished as the \emph{start symbol}, and the set of strings it denotes is the language generated by the grammar. [\ldots]
			\item The productions of a grammar specify the manner in which the terminals and nonterminals can be combined to form strings. Each \emph{production} consists of:
			\begin{itemize}
				\item A nonterminal called the \emph{head} or \emph{left side} of the production; this production defines some of the strings denoted by the head.
				\item{} [\ldots]
				\item A \emph{body} or \emph{right side} consisting of zero or more terminals and nonterminals. The components of the body describe one way in which strings of the nonterminal at the head can be constructed.''
			\end{itemize}
		\end{itemize}
	}
	% LTeX: language=it
	\colchunk{
		\leftskip=1em
		``Una grammatica context-free (grammatica in breve) \`e costituita da \emph{terminali}, \emph{non terminali}, un \emph{simbolo di partenza} e da \emph{produzioni}.

		\begin{itemize}
			\item I \emph{terminali} sono i simboli di base da cui sono formate le stringhe. Il termine "nome del token" \`e sinonimo di "terminale" [\ldots].
			\item I \emph{non terminali} sono variabili sintattiche che denotano insiemi di stringhe. [\ldots]. I non terminali impongono una struttura gerarchica al linguaggio che \`e fondamentale per l'analisi sintattica e la traduzione.
			\item In una grammatica, un non terminale \`e distinto come il \emph{simbolo di partenza} e l'insieme di stringhe che denota \`e il linguaggio generato dalla grammatica. [\ldots]
			\item Le produzioni di una grammatica specificano il modo in cui i terminali e i non terminali possono essere combinati per formare stringhe. Ogni \emph{produzione} \`e costituita da:
			\begin{itemize}
				\item Un non terminale chiamato \emph{testa} o \emph{lato sinistro} della produzione; questa produzione definisce alcune delle stringhe denotate dalla testa.
				\item{} [\ldots]
				\item Un \emph{corpo} o \emph{lato destro} costituito da zero o pi\`u terminali e non terminali. I componenti del corpo descrivono un modo in cui le stringhe del non terminale nella testa possono essere costruite.''
			\end{itemize}
		\end{itemize}
	}
	\colplacechunks
\end{parcolumns}

\`E bene notare che questo tipo di grammatica non \`e capace di descrivere l'intera sintassi del linguaggio. Ad esempio, il requisito che una variabile sia dichiarata prima del suo utilizzo o che il valore assegnatogli sia del tipo corretto non pu\`o essere descritto in questo modo. Questo tipo di controlli \`e compito della fase di analisi semantica.

Vediamo alcuni costrutti sintattici di BugginOut scritti in BNF (\textit{Backus-Naur Form}).

\begin{figure}[H]
	\begin{minted}[breaklines,frame=lines,fontsize=\footnotesize]{text}
<function_parameter> ::= "anon"? "mut"? <Identifier> ":" <Type>
<function_parameters> ::= <function_parameter> ("," <function_parameter>)*

<FunctionDeclarationStatement> ::=
	"fn" <Identifier> "(" <function_parameters>? ")" ":" <Type>
	<BlockExpression>
	\end{minted}
	\label{fig:bugginout-function-declaration}
	\caption{Grammatica per la dichiarazione di una funzione}
\end{figure}

In questo esempio possiamo notare che una dichiarazione di funzione \`e definita, in ordine, da:
\begin{itemize}
	\item la parola chiave \texttt{fn};
	\item un'identificatore (il token \texttt{Identifier});
	\item una parentesi tonda aperta (il token \texttt{LeftParenthesis});
	\item una lista opzionale di parametri definiti dalla regola \raggedright\texttt{<function\_parameters>};
	\item una parentesi tonda chiusa (il token \texttt{RightParenthesis});
	\item due punti (il token \texttt{Colon});
	\item un tipo, definito dalla regola \texttt{<Type>};
	\item un blocco, definito dalla regola \texttt{<BlockExpression>}.
\end{itemize}

A loro volta, i parametri della funzione sono definiti da un parametro di funzione, definito dalla regola \texttt{<function\_parameter>}, seguito da una virgola e da un altro parametro di funzione ripetuti zero o pi\`u volte. \`E importante osservare che la ricorsione \`e destra e non sinistra, il motivo verr\`a approfondito quando si discuter\`a del tipo di approccio utilizzato per l'analisi grammaticale.

Un altro esempio interessante \`e la definizione delle espressioni binarie:
\begin{figure}[H]
	\begin{minted}[breaklines,frame=lines,fontsize=\footnotesize]{text}
<BinaryExpression> ::=
	<Expression>
	("+" | "-" | "*" | "/" | "%" | "<<" | ">>" | "<" | ">" | "<=" | ">=" | "==" | "!=" | "&" | "^" | "|" | "&&" | "||")
	<Expression>
	\end{minted}
	\label{fig:bugginout-binary-expression}
	\caption{Grammatica per le espressioni binarie}
\end{figure}

Il motivo per cui ci interessa analizzare questa regola \`e la sua \emph{ambiguit\`a}, ci\`o significa che esistono pi\`u modi d'interpretare la stessa espressione. Ad esempio, l'espressione \texttt{a + b * c} pu\`o essere interpretata in due modi diversi:
\begin{itemize}
	\item \texttt{(a + b) * c}, in cui l'operazione di somma viene eseguita prima della moltiplicazione;
	\item \texttt{a + (b * c)}, in cui l'operazione di moltiplicazione viene eseguita prima della somma.
\end{itemize}
Citando \cite{alfred2007compilers}:
\begin{parcolumns}[colwidths={1=0.44\textwidth,2=0.44\textwidth},rulebetween=true,nofirstindent=true,sloppy=true]{2}
	% LTeX: language=en_us
	\colchunk{
		\leftskip=1em
		``There are two reasons why we might prefer to use the ambiguous grammar. First, as we shall see, we can easily change the \emph{associativity} and \emph{precedence} of the operators without disturbing the productions or the number of states in the resulting parser.~[\ldots]''
	}
	% LTeX: language=it
	\colchunk{
		\leftskip=1em
		``Ci sono due ragioni per le quali potremmo preferire utilizzare la grammatica ambigua. Per prima cosa, come vedremo, possiamo facilmente cambiare l'\emph{associatività} e \emph{precedenza} degli operatori senza disturbare le produzioni o il numero di stati dell'analizzatore grammaticale.~[\ldots]''
	}
	\colplacechunks
\end{parcolumns}

Chiarito il concetto di grammatica e avendone visto alcuni esempi, \`e possibile passare alla discussione dell'analizzatore grammaticale.

In generale un analizzatore grammaticale pu\`o seguire due principali strategie:
\begin{itemize}
	\item \emph{top-down}, in cui l'analizzatore grammaticale inizia dalla radice dell'albero e scende verso le foglie;
	\item \emph{bottom-up}, in cui l'analizzatore grammaticale inizia dalle foglie dell'albero e risale verso la radice.
\end{itemize}
La strategia scelta per BugginOut \`e quella \emph{top-down} per la sua semplicit\`a e facilit\`a d'uso. Pi\`u specificamente, l'analizzatore lessicale \`e un \textit{predictive parser} senza \textit{backtracking} costruito su una grammatica LL(1).

Questo significa che, a ogni passo, si sceglie la produzione grammaticale corretta utilizzando solo il prossimo token restituito dall'analizzatore lessicale. \`E importante notare che, per costruire un analizzatore grammaticale di questo tipo, \`e necessario che la grammatica sia priva di \emph{ambiguit\`a} e \emph{ricorsioni sinistre}. La prima condizione \`e stata soddisfatta definendo l'associativit\`a e la precedenza degli operatori\footnote{Lo si vede nel dettaglio nel capitolo \ref{chap:architettura-del-compilatore}.} e la seconda scrivendo meticolosamente la grammatica per evitarle.

Durante l'analisi grammaticale, si potrebbe incontrare un errore sintattico se un token incontrato non corrisponde a quello che ci si aspetta nella produzioen. In questo caso l'analizzatore grammaticale genera un errore e termina l'esecuzione. Questo approccio di gestione degli errori \`e il pi\`u semplice e non \`e in grado di recuperare gli errori. Su questo tema si spenderanno alcune parole nella sezione \ref{sec:limiti}.

\section{Analisi semantica}
\label{sec:analisi-semantica}

\section{Generazione del codice}
\label{sec:generazione-del-codice}
