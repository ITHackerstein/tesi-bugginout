% LTeX: language=it

\chapter{Alcuni esempi}
\label{chap:alcuni-esempi}

In questo capitolo, si vedono alcuni esempi di codice scritti in BugginOut, che mostrano le caratteristiche del linguaggio e le sue funzionalità principali.

\mdfdefinestyle{examplestyle}{
	skipabove=0pt,
	innertopmargin=0pt,
	leftmargin=0pt,
	innerleftmargin=0pt,
	skipbelow=0pt,
	innerbottommargin=0pt,
	rightmargin=0pt,
	innerrightmargin=0pt,
	linewidth=0pt
}

\section*{\textit{Hello World}}

\hfill

\begin{mdframed}[style=examplestyle]
	\begin{minted}[breaklines,frame=lines,linenos,fontsize=\footnotesize]{text}
fn main(): void {
  println("Hello, World!");
}
	\end{minted}
\end{mdframed}

\section*{Fibonacci}

\hfill

\begin{mdframed}[style=examplestyle]
	\begin{minted}[breaklines,frame=lines,linenos,fontsize=\footnotesize]{text}
fn main(): void {
  mut a = 0;
  mut b = 1;

  for (b < 1000) {
    println(b);

    var c = a + b;
    a = b;
    b = c;
  }
}
	\end{minted}
\end{mdframed}

\section*{Fattoriale}

\hfill

\begin{mdframed}[style=examplestyle]
	\begin{minted}[breaklines,frame=lines,linenos,fontsize=\footnotesize]{text}
fn factorial(anon n: u64): u64 {
  if (n == 0) {
    1_u64
  } else {
    n * factorial(n - 1)
  }
}

fn main(): void {
  for (i in 0..<10_u64) {
    print(i);
    print("! = ");
    println(factorial(i));
  }
}
	\end{minted}
\end{mdframed}

\section*{Ricerca binaria}

\hfill

\begin{mdframed}[style=examplestyle]
	\begin{minted}[breaklines,frame=lines,linenos,fontsize=\footnotesize]{text}
fn binary_search(array: []i32, element: i32): isize {
  mut low = 0_usize;
  mut high = array.size - 1;
  for (low <= high) {
    mut mid = low + (high - low) / 2;
    if (array[mid] == element) {
      return mid as isize;
    } else if (array[mid] < element) {
      low = mid + 1;
    } else {
      high = mid - 1;
    }
  }

  -1
}

fn main(): void {
  var array = [1, 2, 3, 4, 5, 6, 7, 8, 9];
  var index = binary_search(array, element: 5);

  if (index >= 0) {
    print("Element found at index: ");
    println(index);
  } else {
    println("Element not found in the array");
  }
}
	\end{minted}
\end{mdframed}

\section*{Ordinamento}

\hfill

\begin{mdframed}[style=examplestyle]
	\begin{minted}[breaklines,frame=lines,linenos,fontsize=\footnotesize]{text}
fn bubble_sort(anon array: []mut i32): void {
  for (i in 0..<array.size - 1) {
    for (j in i + 1..<array.size) {
      var tmp = array[i];
      array[i] = array[j];
      array[j] = tmp;
    }
  }
}

fn quick_sort_helper(anon array: []mut i32, anon low: usize, anon high: usize): void {
  var size = high - low;
  if (size < 2) {
    return;
  }

  var pivot = array[low + size / 2];
  mut i = low;
  mut j = high - 1;
  for {
    for (array[i] < pivot) { ++i; }
    for (array[j] > pivot) { --j; }

    if (i >= j) {
      break;
    }

    var tmp = array[i];
    array[i] = array[j];
    array[j] = tmp;
  }

  quick_sort_helper(array, low, i);
  quick_sort_helper(array, i, high);
}

fn quick_sort(anon array: []mut i32): void { quick_sort_helper(array, 0_usize, array.size); }

fn print_array(anon array: []i32): void {
  print('[');
  for (i in 0..<array.size) {
    print(array[i]);
    if (i != array.size - 1) {
      print(", ");
    }
  }
  println(']');
}

fn main(): void {
  var array = [5, 4, 3, 2, 1];
  print("Original array: ");
  print_array(array);

  {
    mut array_copy = array;
    bubble_sort(array_copy);
    print("After bubble sort: ");
    print_array(array_copy);
  }

  {
    mut array_copy = array;
    quick_sort(array_copy);
    print("After quick sort: ");
    print_array(array_copy);
  }
}
	\end{minted}
\end{mdframed}

\section*{Risoluzione del Sudoku con DFS + backtracking}

\hfill

\begin{mdframed}[style=examplestyle]
	\begin{minted}[breaklines,frame=lines,linenos,fontsize=\footnotesize]{text}
fn is_valid_after_move(board: []u8, row: usize, column: usize): bool {
  for (i in 0..<9_usize) {
    if (i != column && board[row * 9 + i] == board[row * 9 + column]) {
      return false;
    }

    if (i != row && board[i * 9 + column] == board[row * 9 + column]) {
      return false;
    }
  }

  var block_row = (row / 3) * 3;
  var block_column = (column / 3) * 3;
  for (i in 0..<3_usize) {
    for (j in 0..<3_usize) {
      if (block_row + i != row && block_column + j != column && board[(block_row + i) * 9 + (block_column + j)] == board[row * 9 + column]) {
        return false;
      }
    }
  }

  true
}

fn solve(board: []mut u8): bool {
  for (row in 0..<9_usize) {
    for (column in 0..<9_usize) {
      if (board[row * 9 + column] != 0) {
        continue;
      }

      for (num in 1..=9_u8) {
        board[row * 9 + column] = num;
        if (is_valid_after_move(board, row, column) && solve(board)) {
          return true;
        }
        board[row * 9 + column] = 0_u8;
      }

      return false;
    }
  }

  true
}

fn main(): void {
  mut board = [
    0_u8, 0_u8, 0_u8, 2_u8, 6_u8, 0_u8, 7_u8, 0_u8, 1_u8,
    6_u8, 8_u8, 0_u8, 0_u8, 7_u8, 0_u8, 0_u8, 9_u8, 0_u8,
    1_u8, 9_u8, 0_u8, 0_u8, 0_u8, 4_u8, 5_u8, 0_u8, 0_u8,
    8_u8, 2_u8, 0_u8, 1_u8, 0_u8, 0_u8, 0_u8, 4_u8, 0_u8,
    0_u8, 0_u8, 4_u8, 6_u8, 0_u8, 2_u8, 9_u8, 0_u8, 0_u8,
    0_u8, 5_u8, 0_u8, 0_u8, 0_u8, 3_u8, 0_u8, 2_u8, 8_u8,
    0_u8, 0_u8, 9_u8, 3_u8, 0_u8, 0_u8, 0_u8, 7_u8, 4_u8,
    0_u8, 4_u8, 0_u8, 0_u8, 5_u8, 0_u8, 0_u8, 3_u8, 6_u8,
    7_u8, 0_u8, 3_u8, 0_u8, 1_u8, 8_u8, 0_u8, 0_u8, 0_u8
  ];

  if (solve(board)) {
    println("Solved Sudoku:");
    for (row in 0..<9) {
      for (column in 0..<9) {
        if (board[row * 9 + column] == 0) {
          print('.');
        } else {
          print(board[row * 9 + column]);
        }
        print(" ");
      }
      print("\n");
    }
  } else {
    println("No solution exists");
  }
}
	\end{minted}
\end{mdframed}

