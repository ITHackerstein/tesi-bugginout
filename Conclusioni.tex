% LTeX: language=it

\chapter*{Conclusioni}
\label{chap:conclusioni}

In questa tesi, ci siamo proposti di analizzare la creazione di un linguaggio di programmazione dalle scelte di design, all'analisi delle fasi teoriche di compilazione e la loro implementazione.

Il risultato \`e un linguaggio semplice e funzionale con, tuttavia, dei limiti. Per prima cosa, il compilatore utilizza solo, ed esclusivamente, \texttt{g++} a causa dell'utilizzo dei \textit{compound statements}. Inoltre, l'utilizzo dell'inferenza \`e limitata alla dichiarazione di variabile di variabili.

In futuro, l'obiettivo \`e di rimuovere tali limiti e di arricchire il linguaggio con funzionalit\`a pi\`u complesse la possibilit\`a di definire tipi personalizzati, di utilizzare caratteri Unicode, l'allocazione dinamica della memoria e la gestione degli errori. Oltre a questo, potrebbe essere anche possibile generare direttamente un eseguibile esplorando, cos\`i, anche le tecniche di generazione di codice intermedio e ottimizzazione evitando di basarsi su un linguaggio e compilatore preesistente.

In conclusione, questa tesi mi ha permesso di approfondire la creazione di un linguaggio di programmazione in tutti i suoi aspetti e di metterli in pratica in maniera concreta.
